\chapter{Giới thiệu}
\newpage

\section{Đặt vấn đề}
Theo ước tính của Tổ chức Y tế thế giới, hàng năm trên thế giới có khoảng 17,5 triệu người tử vong do các bệnh liên quan đến tim mạch và số bệnh nhân tim mạch tích lũy ngày một nhiều. Theo dự báo của Hội Tim mạch Việt Nam, khoảng 20\% dân số nước ta mắc bệnh về tim mạch và tăng huyết áp. Tỷ lệ tăng huyết áp ở những người trẻ từ 25 tuổi đang gia tăng, chiếm 21,5\% tổng số ca mắc bệnh.Tuy nhiên, vẫn còn nhiều người thờ ơ, chủ quan và thiếu quan tâm đến sức khoẻ tim mạch của mình. Theo thống kê của Hội tim mạch Việt Nam, vào năm 1980, tỷ lệ mắc bệnh tim mạch ở tuổi 50 trở lên chỉ ở mức 11\%, thì đến năm 2009, tỷ lệ này lên đến 25\% và độ tuổi mắc từ 22 tuổi trở lên.\cite{baibao}\par

\section{Mục tiêu}
Vì thế việc phát hiện sớm những dấu hiệu bất thường của tim đóng vai trò rất quan trọng giúp giảm nguy cơ mắc các bệnh tim mạch. Trong đó một trong những phương pháp để phát hiện bệnh là đo điện tim. Những để đo điện tim thì một người phải vào các cơ sở y tế với các thiết bị chuyện dụng. Cách làm này tốn khá nhiều thời gian và tiền bạc. Trong khi đó nếu ta có thể thu thập tín hiệu điện tim từ các thiết bị IoT tiếp qua một hệ thống phân loại để phát hiện bất thường ở điện tim một cách nhanh chóng và tương đối chính xác sẽ góp phần giám thiểu nguy cơ mắc bệnh tim mạch. Mục tiêu của để tài là sử dụng một hệ thống phân loại sử dụng các kỹ thuật học sâu (Deep Learning) để phát hiện loại bất thường ở điện tim.

